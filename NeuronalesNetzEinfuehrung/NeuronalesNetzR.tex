\documentclass[]{article}
\usepackage{lmodern}
\usepackage{amssymb,amsmath}
\usepackage{ifxetex,ifluatex}
\usepackage{fixltx2e} % provides \textsubscript
\ifnum 0\ifxetex 1\fi\ifluatex 1\fi=0 % if pdftex
  \usepackage[T1]{fontenc}
  \usepackage[utf8]{inputenc}
\else % if luatex or xelatex
  \ifxetex
    \usepackage{mathspec}
  \else
    \usepackage{fontspec}
  \fi
  \defaultfontfeatures{Ligatures=TeX,Scale=MatchLowercase}
\fi
% use upquote if available, for straight quotes in verbatim environments
\IfFileExists{upquote.sty}{\usepackage{upquote}}{}
% use microtype if available
\IfFileExists{microtype.sty}{%
\usepackage{microtype}
\UseMicrotypeSet[protrusion]{basicmath} % disable protrusion for tt fonts
}{}
\usepackage[margin=1in]{geometry}
\usepackage{hyperref}
\hypersetup{unicode=true,
            pdftitle={NeuronalesNetzinR},
            pdfborder={0 0 0},
            breaklinks=true}
\urlstyle{same}  % don't use monospace font for urls
\usepackage{graphicx,grffile}
\makeatletter
\def\maxwidth{\ifdim\Gin@nat@width>\linewidth\linewidth\else\Gin@nat@width\fi}
\def\maxheight{\ifdim\Gin@nat@height>\textheight\textheight\else\Gin@nat@height\fi}
\makeatother
% Scale images if necessary, so that they will not overflow the page
% margins by default, and it is still possible to overwrite the defaults
% using explicit options in \includegraphics[width, height, ...]{}
\setkeys{Gin}{width=\maxwidth,height=\maxheight,keepaspectratio}
\IfFileExists{parskip.sty}{%
\usepackage{parskip}
}{% else
\setlength{\parindent}{0pt}
\setlength{\parskip}{6pt plus 2pt minus 1pt}
}
\setlength{\emergencystretch}{3em}  % prevent overfull lines
\providecommand{\tightlist}{%
  \setlength{\itemsep}{0pt}\setlength{\parskip}{0pt}}
\setcounter{secnumdepth}{0}
% Redefines (sub)paragraphs to behave more like sections
\ifx\paragraph\undefined\else
\let\oldparagraph\paragraph
\renewcommand{\paragraph}[1]{\oldparagraph{#1}\mbox{}}
\fi
\ifx\subparagraph\undefined\else
\let\oldsubparagraph\subparagraph
\renewcommand{\subparagraph}[1]{\oldsubparagraph{#1}\mbox{}}
\fi

%%% Use protect on footnotes to avoid problems with footnotes in titles
\let\rmarkdownfootnote\footnote%
\def\footnote{\protect\rmarkdownfootnote}

%%% Change title format to be more compact
\usepackage{titling}

% Create subtitle command for use in maketitle
\newcommand{\subtitle}[1]{
  \posttitle{
    \begin{center}\large#1\end{center}
    }
}

\setlength{\droptitle}{-2em}

  \title{NeuronalesNetzinR}
    \pretitle{\vspace{\droptitle}\centering\huge}
  \posttitle{\par}
    \author{}
    \preauthor{}\postauthor{}
    \date{}
    \predate{}\postdate{}
  

\begin{document}
\maketitle

\section{Neuronales Netz Modell in R}\label{neuronales-netz-modell-in-r}

\subsection{Allgemeines}\label{allgemeines}

Ein neuronales Netz (ANN - Artificial Neuronal Network) anhand
bestehender Beispiele zu lernen. ein ANN ist ein
Informationsverarbeitungsmodell, welches vom biologischem Neuronensystem
inspiriert ist. Es besteht aus einer Vielzahl an miteinander verbundenen
Verarbeitungselementen, welche als Neuronen bezeichnet werden. Ein
neuronales Netz ist ein komplexes adaptives (=\textgreater{} interne
Struktur mittels Gewichte und Inputs veränderbar) System.

Neuronale Netze wurden entwickelt um Problemstellungen zu lösen, welche
für Menschen einfach aber für Maschinen schwierig zu lösen sind, zbB.
die Erkennung und Klassifizierungen von Bildern. Solch gelagerte
Problemstellungen werden als Problem der Mustererkennung (``Pattern
Recognition'') eingestuft. Anwendungen dazu sind zB optische
Zeichenerkennung und Erkennung von Objekten.

\subsection{Einführung in neuronale
Netze}\label{einfuhrung-in-neuronale-netze}

Ein Neuron ist eine simple Zelle innerhalb eines neuronalen Netzes(NN)
welches einen Input entgegennimmt, diesen verarbeitet und diesen
ausgibt.

Der Algirthmus neuronaler Netze ist inspiriert von der Funktionsweise
des menschlichen Gehirns um bestimmte Aufgaben zu erledigen. Ein NN
führt Berechnungen mittels Lernprozesse durch. Das neuronale Netz
besteht aus verbundenen Ein- und Ausgabeeinheiten, wobei jeder
Verbindung eine Gewichtung zugeordnet ist. Während des Lernprozesses
lernt das neuroanle Netz durch Anpassung dieser Gewichtungen um zB. die
richtige Klassenbezeichnung von EIngabedaten zu ermitteln.


\end{document}
